\Header{mhisn}{~~function to do ... ~~}
\keyword{~kwd1}{mhisn}
\keyword{~kwd2}{mhisn}
\begin{Description}\relax
~~ A concise (1-5 lines) description of what the function does. ~~\end{Description}
\begin{Usage}
\begin{verbatim}
mhisn(dat, mcmc, tau=1, theta0, counter=T)
\end{verbatim}
\end{Usage}
\begin{Arguments}
\begin{ldescription}
\item[\code{dat}] ~~Describe \code{dat} here~~ 
\item[\code{mcmc}] ~~Describe \code{mcmc} here~~ 
\item[\code{tau}] ~~Describe \code{tau} here~~ 
\item[\code{theta0}] ~~Describe \code{theta0} here~~ 
\item[\code{counter}] ~~Describe \code{counter} here~~ 
\end{ldescription}
\end{Arguments}
\begin{Details}\relax
~~ If necessary, more details than the __description__  above ~~\end{Details}
\begin{Value}
~Describe the value returned
If it is a LIST, use
\begin{ldescription}
\item[\code{comp1 }] Description of `comp1'
\item[\code{comp2 }] Description of `comp2'
\end{ldescription}

...\end{Value}
\begin{Section}{WARNING}
....\end{Section}
\begin{Note}\relax
~~further notes~~\end{Note}
\begin{Author}\relax
~~who you are~~\end{Author}
\begin{References}\relax
~put references to the literature/web site here ~\end{References}
\begin{SeeAlso}\relax
~~objects to SEE ALSO as \code{\Link{\textasciitilde{}\textasciitilde{}fun\textasciitilde{}\textasciitilde{}}}, ~~~\end{SeeAlso}
\begin{Examples}
\begin{ExampleCode}
##---- Should be DIRECTLY executable !! ----
##-- ==>  Define data, use random,
##--         or do  help(data=index)  for the standard data sets.

## The function is currently defined as
function(dat,mcmc,tau=1,theta0,counter=T)
 {
  # dat = binary matrix, ntumor x nswitch, holding CGH profiles
  # mcmc = list with driving instructions
  #        nskip  = size of stretch between saved states
  #        nsave  = number of saved states
  #        eps    = window for parameter update
  #        pup    = proportion of scans that have parameter  updates
  #        nmove  = number of switches moved in 1st network update
  #        nperm  = number of switches permuted in 2nd network update
  #        padI =  proportion of switches to change in first AD update
  #        padII (<1/2) = chance of full activation / deactivation in 2nd AD
  #
  # tau > 0.      The cluster prior hyperparameter
  # theta0          Starting values for alpha and beta
        
  # Initial state (a biased prior draw, keeping the initial network small)
  nswitch <- ncol(dat)
  nprob <- runif(1,min=(3/4), max=1 )
  null <- sample( c(T,F), replace=T, size=nswitch, prob=c(nprob,1-nprob) ) 
  partition <- rpolya(nswitch,tau)
  theta <- theta0

  # Evaluate the loglikelihood and logprior of starting state
  loglik <- disn(theta, dat, partition, null)
  logpri <- dpolya1(partition,null,tau)

  # Create objects to save MCMC output.
  pup <- mcmc$pup
  nperm <- mcmc$nperm  # number of switches permuted in Update II
  nmove <- mcmc$nmove
  padI <- mcmc$padI
  padII   <- mcmc$padII
  nscan <- mcmc$nsave*mcmc$nskip

  psave <- matrix(NA,mcmc$nsave,nswitch)     # saved partitions
  liksave <- numeric(mcmc$nsave)             # saved log-likelihood 
  prisave <- numeric(mcmc$nsave)             # saved log-prior
  thsave <- matrix(NA,mcmc$nsave,2)          # saved parameter values
  dimnames(thsave) <- list( NULL, c("alpha","beta") )
  acrate <- rep(0,5)              # acceptance rates
  names(acrate) <- c("change path","swap","act./deact. I",
                        "act./deact. II", "params")

  ####################################################################
  # Run MCMC.

  skipcount <- 0
  isave <- 1
  for(iscan in 1:nscan)
   {
    skipcount <- skipcount+1
    ##################################################################
    #
    # Propose a new network configuration (4 move types)
    #
    ##################################################################
    # 
    # Update I: change the partition structure

    ii <- sample( (1:nswitch), size=nmove,replace=F ) #source switches
    maintain <- rep(T,nswitch)
    maintain[ii] <- F
    # replace from the conditional prior (so prior cancels in MH ratio)
    proposal <- partition
    proposal[ii] <- runif(nmove)  # possible new labels
    existing <- partition[maintain]
    for( i in 1:nmove )
      {
       pwild <- tau/(tau+nswitch-nmove+i-1)
       if( runif(1) > pwild ) # don't take the new label
         { proposal[(ii[i])] <- sample( existing, size=1 ) }
       existing <- c(existing, proposal[(ii[i])] )
      }
    # Check whether or not to accept
    ploglik <- disn(theta, dat, proposal, null )
    logmh <- ploglik-loglik
    ok <- !is.na(logmh)
    if( ok )
     {
      accept <- ( log(runif(1)) <= logmh )
      if( accept )
       {
        partition <- proposal
        loglik <-  ploglik
        logpri <- dpolya1(partition,null,tau)
        acrate[1] <- acrate[1]+1
       }
     }
    ##################################################################
    #
    # Update II: shuffle the partition

    ii <- sample( (1:nswitch), size=nperm, replace=F)
    jj <- sample(ii)  # permute these
    proposal <- partition
    proposal[ii] <- partition[jj]
    # Check whether or not to accept if proposal != partition
    if( any( proposal != partition ) )
     {
      ploglik <- disn(theta, dat, proposal, null )
      logmh <- ploglik-loglik
      ok <- !is.na(logmh)
      if( ok )
       {
        accept <- ( log(runif(1)) <= logmh )
        if( accept )
          {
            partition <- proposal
            loglik <-  ploglik
            acrate[2] <- acrate[2]+1
           }
         }
       }
    ##################################################################
    #
    # Update method III:
    #   Change the activation status (active/null) of a random set
    change <- ( runif(nswitch) <= padI )
    if( any(change) )
     {
      nullstar <- null
      nullstar[change] <- !null[change]
      ploglik <- disn(theta, dat, partition, nullstar )
      plogpri <- dpolya1(partition,nullstar,tau)
      logmh <- ploglik-loglik+plogpri-logpri
      ok <- !is.na(logmh)
      if( ok )
       {
        accept <- ( log(runif(1)) <= logmh )
        if( accept )
         {
          null <- nullstar
          logpri <- plogpri
          loglik <- ploglik
          acrate[3] <- acrate[3]+1
         }
       }
     }
    ##################################################################
    # Update method IV:
    #   Change the activation status of a whole path (of length >= 2)
     
    path <- sample( partition, size=1 )
    ind <- (partition == path) 
    mm <- sum(ind)
    if( mm > 1 )  # attempt only if path at least 2 switches long
     {
      nullpart <- null[ind]
      mm0 <- sum( nullpart )
      lqnum <- log( padII )   # log( q(state* , state ) )
      if( mm0 > 0 & mm0 < mm )
       {
        lqnum <- log( 1 - 2*padII ) - log( 2^mm - 2 ) 
       }
      xi <- runif(1)
      if( xi <= padII )  # fully activate path (make sure padII < 1/2)
       {
         nullnew <- rep(F,mm)
         lqden <- log( padII )
       }
      else if( xi <= 2*padII )  # fully deactivate path
       {
         nullnew <- rep(T,mm)
         lqden <- log( padII )
       }
      else  # sample one of the 2^mm - 2 mixed arrangements
       {
         nullnew <- mixsample( mm )
         lqden <- log( 1 - 2*padII ) - log( 2^mm - 2 ) 
       } 
      nullstar <-  null
      nullstar[ind] <- nullnew
      ploglik <- disn(theta, dat, partition, nullstar )
      plogpri <- dpolya1(partition,nullstar,tau)
      logmh <- ploglik-loglik+plogpri-logpri+lqnum - lqden
      ok <- !is.na(logmh)
      if( ok )
       {
        accept <- ( log(runif(1)) <= logmh )
        if( accept )
         {
          null <- nullstar
          logpri <- plogpri
          loglik <- ploglik
          acrate[4] <- acrate[4]+1
         }
       }
     }
    ###################################################################
    # Update the instability parameters theta=(alpha,beta) on some scans
    # symmetric random walk proposal
 
    if( runif(1) < pup )
     { 
      thetastar <- theta + mcmc$eps*runif(2,min=(-1),max=(1))
      # enforce a prior ordering alpha >= beta
      aok <- (thetastar[2]> 0)&(thetastar[1]<1)&(thetastar[1]>=thetastar[2])
      if( aok )
       {
        # Check whether or not to accept the proposal 
        # (The partition prior is not a factor.) 
  
        ploglik <- disn(thetastar, dat, partition, null )
        logmh <- ploglik - loglik
        ok <- !is.na(logmh)
        if(ok)
         {
          accept <- ( log(runif(1)) <= logmh )
          if( accept )
           {
            theta <- thetastar 
            loglik <- ploglik
            acrate[5] <- acrate[5]+1
           }
         }
       }
     }
    #####################################################################
    #
    # Store summary statistics periodically..
    if( skipcount == mcmc$nskip )
     {
      skipcount <- 0
      spart <- partition
      spart[null] <- 0
      psave[isave,] <- spart
      liksave[isave] <- loglik
      prisave[isave] <- dpolya2(spart,(spart==0),tau)
      thsave[isave,] <- theta
      if( counter ){ print( isave ) }
      isave <- isave+1
     }
   }

  # Normalize acceptance rate and store results.
  acrate[3] <- acrate[3]/( 1-(1-padI)^nswitch )
  acrate[5] <- acrate[5]/pup
  acrate <- acrate/( nscan )
  
  results <- list( partitions=psave, loglik=liksave, logpri=prisave, 
                   acrate=acrate, data=dat, theta=thsave, 
                   mcmc=mcmc, tau=tau )

  return( results )
 }
\end{ExampleCode}
\end{Examples}

