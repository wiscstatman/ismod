\Header{dpolya1}{~~function to do ... ~~}
\keyword{~kwd1}{dpolya1}
\keyword{~kwd2}{dpolya1}
\begin{Description}\relax
~~ A concise (1-5 lines) description of what the function does. ~~\end{Description}
\begin{Usage}
\begin{verbatim}
dpolya1(partition, null, tau)
\end{verbatim}
\end{Usage}
\begin{Arguments}
\begin{ldescription}
\item[\code{partition}] ~~Describe \code{partition} here~~ 
\item[\code{null}] ~~Describe \code{null} here~~ 
\item[\code{tau}] ~~Describe \code{tau} here~~ 
\end{ldescription}
\end{Arguments}
\begin{Details}\relax
~~ If necessary, more details than the __description__  above ~~\end{Details}
\begin{Value}
~Describe the value returned
If it is a LIST, use
\begin{ldescription}
\item[\code{comp1 }] Description of `comp1'
\item[\code{comp2 }] Description of `comp2'
\end{ldescription}

...\end{Value}
\begin{Section}{WARNING}
....\end{Section}
\begin{Note}\relax
~~further notes~~\end{Note}
\begin{Author}\relax
~~who you are~~\end{Author}
\begin{References}\relax
~put references to the literature/web site here ~\end{References}
\begin{SeeAlso}\relax
~~objects to SEE ALSO as \code{\Link{\textasciitilde{}\textasciitilde{}fun\textasciitilde{}\textasciitilde{}}}, ~~~\end{SeeAlso}
\begin{Examples}
\begin{ExampleCode}
##---- Should be DIRECTLY executable !! ----
##-- ==>  Define data, use random,
##--         or do  help(data=index)  for the standard data sets.

## The function is currently defined as
function(partition,null,tau)
 {
  #####################################################################
  #
  # Evaluate the log prior mass of a Double Polya Prior-augmented state.
  #
  # This is the prior for an augmented-state partition in ISN model
  # (Used in the data-augmentation part of the MCMC algorithm.)
  #
  #
  # partition = nswitch vector holding distinct path labels
  # null = logical vector indicating active (i.e. relevant) switches
  # tau = clustering parameter: bigger <-> more paths
  #
  #####################################################################

  nswitch <- length(partition)

  # The partition componenent
  incidence <- outer( partition, unique(partition), "==" )
  sizes <- t(incidence) 
  npath <- length(sizes)
  logp <-  npath*log(tau) + sum( lgamma(sizes) ) 
  logp <- logp + lgamma(tau)-lgamma(nswitch+tau)

  # The activation component
  nnull <- sum( null )
  logp <- logp+lgamma(nnull+1)+lgamma(nswitch-nnull+1)-lgamma(nswitch+2)
  return( logp )
 }
\end{ExampleCode}
\end{Examples}

